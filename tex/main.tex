\documentclass[a4paper, 14pt]{report}

%% PACKAGES %%%%%%%%%%%%%%%%%%%%%%%%%%%%%%%%%%%%%%%%%%%%%%%%

\usepackage{amsmath}  % math
\usepackage{graphicx} % embedding images
\usepackage{enumitem} % lists
\usepackage{hyperref} % urls
\usepackage{logicproof} % fitch style proofs

\usepackage{geometry} % margins
\geometry{margin=0.9in} % make the margin smaller

\graphicspath{{../img/}{./img/}}

%% END OF PACKAGES %%%%%%%%%%%%%%%%%%%%%%%%%%%%%%%%%%%%%%%%%%

\author{%
	Norbert Logiewa \\
	\small{nl253}}

\date{December 2017}

\title{%
	Assessment 2 \\
	\Huge{%
		Context Free Grammars \\
		and \\
		Turing Machines}}

% MACROS %%%%%%%%%%%%%%%%%%%%%%%%%%%%%%%%%%%%%%%%%%%%%%%%%%%%%

\newcommand{\topic}[1]{%
	\pagebreak
	\section*{%
		\begin{center} \huge{#1} \end{center}}}

\newcommand{\centeredimg}[1]{%
	\begin{figure}[h]
		\begin{center}
			\includegraphics[height=7cm]{#1}
		\end{center}
	\end{figure}} 

\newcommand{\answer}[1]{%
	\begin{flushleft}
		\textbf{Answer}:\\
		#1
	\end{flushleft}}

\newcommand{\question}[1]{\subsection*{#1}}
\newcommand{\task}[1]{%
	\begin{flushleft}
		\textbf{Task:}\\ 
		#1
	\end{flushleft}}

% END OF MACROS %%%%%%%%%%%%%%%%%%%%%%%%%%%%%%%%%%%%%%%%%%%%%%%

\begin{document}

\maketitle

\topic{Context Free Grammars}

\question{1. Consider the language}

\begin{enumerate}		

	\item Give a word that is in the language and a word that is not in the language

		\answer{%
			not in the language: kkk \\
			in the language: 		 aabbbbcc
		}

	\item Give a context-free grammar for the language above.

		\answer{%
			---
		}

	\item Use the Cocke-Younger-Kasami algorithm to determine whether abbaa is a word
		of the language of the following grammar. Give the table. State in one sentence
		whether the word is a word of the language of the grammar and how you obtain
		this conclusion from the table.

		\answer{%

			\begin{center}
				\begin{tabular}{ |l|c|c|c|c|c| } 
					\hline
					5 & x & --- & --- & --- & --- \\
					4 & x & x & --- & --- & --- \\
					3 & x & x & x & --- & --- \\
					2 & x & x & x & x & --- \\
					1 & A & B & B & A & A \\
					\hline
					-- & a & b & b & a & a \\
					\hline
				\end{tabular}
			\end{center}
		}

	\item Give a parse tree for the word abba with respect to the grammar above (for part $c)$)

		\answer{%
			---
		}

	\item What is $FIRST(SS)$ with respect to the grammar above (for part $c)$)

		\answer{%
			---
		}

\end{enumerate}		


\question{2. Consider the following two context-free grammars}

\begin{enumerate}		

	\item Draw two different parse trees for the word cacbc and the grammar $G1$


		\answer{%
			---
		}

	\item Give the $LOOKAHEAD$ set for every rule of grammar $G2$


		\answer{%
			---
		}

	\item Is the grammar $G2$ $LL(1)$?


		\answer{%
			---
		}

	\item Give the set of nullable nonterminals for the grammar $G2$

		\answer{%
			---
		}

	\item Give the context-free grammar that you obtain from replacing all $ε$-rules in grammar $G2$


		\answer{%
			---
		}

\end{enumerate}		


\topic{Turing Machines}

\question{Consider the following Turing machine with input alphabet \{$a$, $b$\} and tape alphabet \{$a$, $b$, $\_$\}}

\begin{enumerate}		

	\item Give computations for the words ab and bb. State for each word whether the
		machine accepts it, rejects it or loops. If the machine loops, then give the first
		five configurations of the computation.

		\answer{%
			---
		}

	\item Draw a Turing machine that decides the language of all words over the alphabet \{$a$, $b$\} that have an odd number of a’s and an odd number of b’s.

		\answer{%
			---
		}

\end{enumerate}		

\pagebreak

% \section*{References}

% \begin{flushleft}
% \begin{itemize}[noitemsep]
% \end{itemize}		
% \end{flushleft}

\end{document}
%% vim: foldmethod=marker conceallevel=0:
