\documentclass[a4paper, 14pt]{report}

%% PACKAGES %%%%%%%%%%%%%%%%%%%%%%%%%%%%%%%%%%%%%%%%%%%%%%%%

\usepackage{amsmath}  % math
\usepackage{graphicx} % embedding images
\usepackage{enumitem} % lists
\usepackage{multicol} % columns
\usepackage{hyperref} % urls

\usepackage{geometry} % margins
\geometry{margin=0.9in} % make the margin smaller

\graphicspath{{../img/}{./img/}}

%% END OF PACKAGES %%%%%%%%%%%%%%%%%%%%%%%%%%%%%%%%%%%%%%%%%%

\author{%
	Norbert Logiewa \\
\small{nl253}}

\date{December 2017}

\title{%
	Assessment 2 \\
	\Huge{%
		Context Free Grammars \\
		and \\
Turing Machines}}

% MACROS %%%%%%%%%%%%%%%%%%%%%%%%%%%%%%%%%%%%%%%%%%%%%%%%%%%%%

\newcommand{\topic}[1]{%
	\pagebreak
	\section*{%
\begin{center} \huge{#1} \end{center}}}

\newcommand{\centeredimg}[1]{%
	\begin{figure}[h]
		\begin{center}
			\includegraphics[height=7cm]{#1}
		\end{center}
\end{figure}} 

\newcommand{\answer}[1]{%
	\textbf{Answer}:\\
#1}

\newcommand{\question}[1]{\subsection*{#1}}

\newcommand{\task}[1]{%
	\begin{flushleft}

		\textbf{Task:}\\ 
		#1
\end{flushleft}}

% END OF MACROS %%%%%%%%%%%%%%%%%%%%%%%%%%%%%%%%%%%%%%%%%%%%%%%

\begin{document}

\maketitle

\topic{Context Free Grammars}

\question{1. Consider the language}

\begin{enumerate}[label=(\alph*)]
	\setlength\itemsep{2em}

\item Give a word that is in the language $\{a^i b^k c^m \ |\ i \geq 0, k \geq 0, m \geq 0, k = i + m \}$ and a word that is not in the language

	\answer{%
		a word not in the language:  aaaaaaaaa \\
		a word in the language: 		 aabbbbcc
	}

\item Give a context-free grammar for the language above.

	\answer{%
		$S \rightarrow B\ |\ a S c$ \\
		$B \rightarrow \epsilon\ |\ b B c$ \\
	}

\item Use the CYK algorithm to determine whether \textbf{abbaa} is a word
	of the language of the following grammar. Give the table. State in one sentence
	whether the word is a word of the language of the grammar and how you obtain
	this conclusion from the table.

	\[ S \Rightarrow AX\ |\ BY\ |\ SS\ |\ BA \]
	\[ X \Rightarrow AS \]
	\[ Y \Rightarrow BS \]
	\[ A \Rightarrow a \]
	\[ B \Rightarrow b \]

	\answer{%
		\begin{center}
			\begin{tabular}{|l|c|c|c|c|c|}
				\hline
				5 & -- & --- & --- & --- & --- \\
				4 & -- & -- & --- & --- & --- \\
				3 & -- & Y & -- & --- & --- \\
				2 & -- & -- & S & -- & --- \\
				1 & A & B & B & A & A \\
				\hline
				-- & a & b & b & a & a \\
				\hline
			\end{tabular}
		\end{center}
		\textbf{Explanation}: \\
		It's not, there is no way to parse it as $S$
		doesn't appear in the top row.}


\item Give a parse tree for the word \textbf{aaba} with respect to the grammar above (for part $c)$)

	\answer{\centeredimg{parse_tree.jpg}}

	\pagebreak


\item What is $FIRST(SS)$ with respect to the grammar above (for part $c)$)

	\answer{$FIRST(SS)$ = \{$a$, $b$\}}
\end{enumerate}		

\question{2. Consider the following two context-free grammars}

\begin{multicols}{2}
	\begin{center} \textbf{\\$G_{2}$} \end{center}

	\[ S \Rightarrow DAd \]
	\[ A \Rightarrow aS\ |\ \epsilon \]
	\[ B \Rightarrow bD\ |\ \epsilon \]
	\[ D \Rightarrow cB \]

	\begin{center} \textbf{\\\\$G_{1}$} \end{center}

	\[ S \Rightarrow SaS\ |\ SbS\ |\ c \] \\
\end{multicols}

\begin{enumerate}[label=(\alph*)]
	\setlength\itemsep{2em}

\item Draw two different parse trees for the word cacbc and the grammar $G_{1}$

	\answer{\centeredimg{parse_trees.jpg}}

\item Give the $LOOKAHEAD$ set for every rule of grammar $G_{2}$
	% % Definition: LOOKAHEAD(X -> α) and LOOKAHEAD(X):
	% % LOOKAHEAD(X -> α) = FIRST(α) U FOLLOW(X), if NULLABLE(α)
	% % LOOKAHEAD(X -> α) = FIRST(α), if not NULLABLE(α)
	% % LOOKAHEAD(X) = LOOKAHEAD(X -> α) U LOOKAHEAD(X -> β) U LOOKAHEAD(X -> γ) 
	% LOOKAHEAD(A ➞ α) = { a | α ⇒* aw} ∪ (if α ⇒* ℇ then Follow(A) else ∅)

	\answer{%
		\\
		\begin{tabular}{ |c|c|c|c|c| }
			\hline
			rule $R \Rightarrow t$ & $NULLABLE(R)$ & $FIRST(t)$ & $FOLLOW(R)$ & $LOOKAHEAD(R)$ \\
			\hline
			\[ S \Rightarrow DAd \] & $false$ & $\{c\}$ & $\{d\}$ & $\{c\} $ \\
			\hline
			\[ A \Rightarrow aS\ |\ \epsilon \] & $true$  & $\{a, \epsilon\}$ & $\{d\}$ & $\{a, d, \epsilon\}$  \\
			\hline 
			\[ B \Rightarrow bD\ |\ \epsilon \] & $true$ & $\{b, \epsilon\}$ & $\{a\}$ & $\{a, b, \epsilon\}$ \\ 
			\hline
			\[ D \Rightarrow cB \] &$false$ & $\{c\}$ & $\{a\}$ & $\{c\}$ \\
			\hline
	\end{tabular}}

\item Is the grammar $G_{2}$ $LL(1)$?

	\answer{No it's not, there are overlapping lookahead sets.}

\item Give the set of nullable non-terminals for the grammar $G_{2}$

	\answer{$\{A, B, D, S\}$}

\item Give the context-free grammar that you obtain from replacing all $\epsilon$-rules in grammar $G_{2}$

	\answer{%

		\begin{multicols}{2}

			\begin{center} 

				\textbf{G_{2}} 

				\[ S \Rightarrow DAd \]
				\[ A \Rightarrow aS\ |\ \epsilon \]
				\[ B \Rightarrow bD\ |\ \epsilon \]
				\[ D \Rightarrow cB \]

			\end{center}

			\begin{center} 

				\textbf{after replacing $\epsilon$ rules} 

				\[ S \Rightarrow DAd\ |\ Dd\ |\ Ad\ |\ d \]
				\[ A \Rightarrow aS\ |\ a  \]
				\[ B \Rightarrow bD\ |\ b \]
				\[ D \Rightarrow cB\ |\ c \]

			\end{center}

		\end{multicols}
	}
\end{enumerate}		

\topic{Turing Machines}

\question{Consider the following Turing machine with input alphabet \{$a$, $b$\} and tape alphabet $\{a, b, \_\}$}

\begin{enumerate}[label=(\alph*)]
	\setlength\itemsep{2em}

\item Give computations for the words \textbf{ab} and \textbf{bb}. 
	State for each word whether the machine accepts it, rejects it or loops. 
	If the machine loops, then give the first five configurations of the computation.

	\answer{%
		\begin{center}
			\begin{tabular}{ |c|l|c| } 
				\hline
				input word & computation & outcome \\
				\hline
				ab & $a/a/R$ \hspace{0.1cm} \vdash \hspace{0.2cm} $b/b/R$ \hspace{0.2cm} \vdash \hspace{0.2cm} $\_/\_/R$ \hspace{0.2cm} \vdash \hspace{0.2cm} $\_/\_/R$ & reject \\
				\hline
				bb & $b/b/R$ \hspace{0.2cm} \vdash \hspace{0.2cm} $a/a/R$ \hspace{0.2cm} \vdash \hspace{0.2cm} $b/b/R$ \hspace{0.2cm} \vdash \hspace{0.2cm} $b/b/R$ \hspace{0.2cm} \vdash \hspace{0.2cm} $b/b/R$ & loop \\
				\hline
			\end{tabular}
	\end{center}}

\item Draw a Turing machine that decides the language of all words
	over the alphabet \{$a$, $b$\} that have an odd number of $a$s and
	an odd number of $b$s.

	\answer{}

\end{enumerate}		

\pagebreak

\end{document}
%% vim: foldmethod=marker conceallevel=0:foldmethod=indent:
